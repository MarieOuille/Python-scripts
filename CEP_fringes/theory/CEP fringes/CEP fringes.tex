
\documentclass[a4paper,twoside,10pt]{report}
%	Paper Size: a4paper / a5paper / b5paper / letterpaper / legalpaper / executivepaper
% Duplex: oneside / twoside
% Base Font Size: 10pt / 11pt / 12pt


\usepackage[USenglish]{babel} %francais, polish, spanish, ...
\usepackage[T1]{fontenc}
\usepackage[ansinew]{inputenc}

\usepackage{lmodern} %Type1-font for non-english texts and characters

\usepackage{graphicx} %%For loading graphic files

\usepackage{amsmath}
\usepackage{amsthm}
\usepackage{amsfonts}


\begin{document}

\pagestyle{empty} %No headings for the first pages.



\title{Simulating CEP fringes for Salle Noire 2.0}
\author{Marie Ouill�}
%\date{} %%If commented, the current date is used.

\maketitle



\pagestyle{plain} %Now display headings: headings / fancy / ...

\chapter{Gaussian spectra}\label{hints}

In this chapter we consider the laser pulses to have a gaussian spectral shape and a flat spectral phase, except for a small amount of GDD introduced to simulate the exit window of the compression chamber.

\section{Fundamental field}\label{umlauts}

The fundamental electric field is expressed by :
\\

$E_{fund}(\omega) = e^ {- \frac{a^2}{2}(\omega - \omega_0)^2}  e^{ -i\Phi(\omega) }$\\ 

with $\Phi(\omega) = k_{00} + k_{10}(\omega - \omega_0) +  \frac{k_{20}}{2!}(\omega - \omega_0)^2 +  \frac{k_{30}}{3!}(\omega - \omega_0)^3 +   \frac{k_{40}}{4!}(\omega - \omega_0)^4 $\\

$\omega_0$ is the central frequency and corresponds to 790nm. 
$k_{00}$ is the CEP.
$k_{20}$ is the GDD or 'chirp'.
\\

\section{SHG field}\label{references}

The fringeezz analyzes the CEP fringes in the spectrum around 480nm.\\

We use a BBO crystal to frequency double the red components. Two photons at 960nm combine to form one photon at 480nm.\\

Let's assume the SHG field is a gaussian centered around $\omega_{0, blue}$, corresponding to 480nm : \\


$E_{SHG}(\omega) = \sqrt{att}  \ e^ {- \frac{a_{blue}^2}{2}(\omega - \omega_{0, blue})^2}  e^{ -i2\Phi(\omega /2) }$ \\


This wave has twice the spectral phase and has a width $a_{blue}$ which depends on the BBO thickness and the fundamental spectral intensity.\\

att is an attenuation factor representing the SHG efficiency and the attenuation by the polarizing beam splitter in the f-2f interferometer, used to maximize the contrast of the fringes.\\

I'm sure we can do better than this, not very clear so far why the SHG field should be expressed like that.


\section{CEP fringes}\label{dividing}
 
The total detected spectral intensity is expressed by : \\

$ I_{tot}(\omega) = |E_{fund}(\omega) + E_{SHG}(\omega)|^2 $\\


The spectral modulations, called 'CEP fringes', are given by the term $\Delta\Phi (\omega) = \Phi_{fund}(\omega) - \Phi_{SHG}(\omega) = \Phi_{fund}(\omega) - 2\Phi_{fund}(\omega/2)$ and more precisely the cosine of $\Delta\Phi$.







\chapter{Measured spectrum and spectral phase (using d-scan measurements)}\label{hints}




\end{document}

